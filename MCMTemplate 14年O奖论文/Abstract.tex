%\begin{flushleft}
\begin{abstract}
Our goal is a model that can evaluate the performance of the keep-right-except-to-pass rule and other alternatives by simulating the traffic flow on the freeway. We construct models to analyze five influencing factors. Then we integrate multiple criteria to judge the performance of nine rules using a fuzzy synthetic evaluation(FSE).

Our basic lane-changing model focuses on the behavior of a specific vehicle on the freeway. We carefully examine the vehicle��s lane-changing behavior, an essential component of overtaking.

We extend our model with a cellular-automaton-based approach. We assume that the drivers will change the lane with a probability if the trigger and safety conditions are satisfied. Using periodic boundary conditions, we seek to simulate a section of a long freeway, which is hardly influenced by real boundary conditions. In addition, we can accurately control the occupancy of the freeway. We can simulate the traffic flow under several conditions by varying \textsl{the number of lanes, maximum speed limit, minimum speed limit and signaling behavior}.

Four other basic rules such as free-overtaking rule are examined by revising the laws governing the cells in the cellular automaton. Then we design five improved rules based on the basic rules attempting to obtain an optimal rule.

We choose flow rate, average speed as traffic flow criteria, sharp braking frequency as a safety criterion and satisfaction and standard deviation of speed as experience criteria. Then we use a fuzzy synthetic evaluation technique to integrate these criteria to determine the performance of each rule. We find that in a light traffic case, \textsl{a partial-assigned-lane-and-keep-right rule} performs the best while in a heavy traffic situation, \textsl{a different-speed-limit-on-each-lane rule} is preferred.

We change the probability of lane-changing to adjust our model to a country like Great Britain. Moreover, we change that parameter to simulate a freeway fully controlled by an intelligent system and observe small deviations.

Additionally, we refine our extended model considering the ramps. We adopt open boundary conditions and assume that the vehicles flowing in are Poisson-distributed. Finally, we change parameters to analyze freeways with ramps under different conditions.

\end{abstract}
